\documentclass[12pt,a4paper]{article}

\usepackage{geometry}
\geometry{margin=1in}
\usepackage{graphicx}   % For figures and logo
\usepackage{hyperref}   % For clickable links
\usepackage{booktabs}   % For better tables
\usepackage{amsmath}    % For equations
\usepackage{setspace}   % Line spacing
\usepackage{enumitem}   % Custom lists
\usepackage{fancyhdr}   % For custom headers/footers
\usepackage{tabularx}
\usepackage{float}  
\usepackage{caption}

% -------------------------------
% Header & Footer setup
% -------------------------------
\pagestyle{fancy}
\fancyhf{} % Clear all header and footer fields

% ULAB logo on the right side of header
\fancyhead[R]{\includegraphics[height=1.2cm]{CommonAssets/ULAB Logo.PNG}}  

% Adjust header spacing
\setlength{\headsep}{1.5cm}

% Page number on the right footer
\fancyfoot[R]{Page \thepage}

% -------------------------------






\title{\textbf{Project Report : Agriculture Crop Yield}\\
STA 2101: Statistics \& Probability}
\author{Student Name : Mahinur Rahman Mahin \\
Student ID : 242014165 \\
University of Liberal Arts Bangladesh (ULAB)}
\date{\today}

\begin{document}

\maketitle
\onehalfspacing

\begin{abstract}
This document is the course project report for STA 2101.This project analyzes by the link of "Agriculture crope Yield".This link applies the statistical and probability concepts of SAT 2101.Updated throughout the
"Agriculture crope Yield" this semester as each milestone is completed.
\end{abstract}

\tableofcontents
\newpage










% -------------------------------


\section{Milestone 1: Dataset Selection}
\begin{itemize}
    \item \textbf{Dataset Name: Agriculture Crop Yield} %[Insert dataset name here]
    \item \textbf{Dataset URL:} \\                \url{https://www.kaggle.com/datasets/samuelotiattakorah/agriculture-crop-yield}  
    \item \textbf{Description:} Rice is the primary food for half of the people in the world. It is also known as staple food in Bangladesh. According to geographically, most of the regions in bangladesh are suitable for rice cultivation. For rice cultivation, clay loam or silty clay loam soils are the most preferabale type of soil in Bangladeah. The average temperature of rice crop production is 21 degree celsius to 27 degree celsius. Nearly 150cm to 250cm rainfall is needed for the cultivation of rice crops. \\\
    Fertilizer and irrigation are used in rice production. \\\
    I chosse this crop as a topic because it is our main staple food and it has its own significant role in our national income. 
\end{itemize}

% -------------------------------





















































\section{Milestone 4: Probability Distributions}
Identify probability distributions in your dataset. Perform fitting, plots, and discuss results.

% -------------------------------
\section{Milestone 5: Hypothesis Testing}
State hypotheses, perform tests, and report conclusions.

% -------------------------------
\section{Milestone 6: Regression Analysis}
Fit regression models, explain coefficients, and evaluate model fit.

% -------------------------------
\section{Milestone 7--12: Further Analysis}
Continue documenting each milestone here as instructed in class.

% -------------------------------
\section{Final Conclusion}
Summarize the overall findings of your project. Mention challenges, learning outcomes, and possible future work.

\newpage
\section*{References}
List your references here in proper citation format. If you prefer, you may use BibTeX.

\end{document}
