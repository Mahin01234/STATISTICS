\documentclass[12pt,a4paper]{article}

\usepackage{geometry}
\geometry{margin=1in}
\usepackage{graphicx}   % For figures and logo
\usepackage{hyperref}   % For clickable links
\usepackage{booktabs}   % For better tables
\usepackage{amsmath}    % For equations
\usepackage{setspace}   % Line spacing
\usepackage{enumitem}   % Custom lists
\usepackage{fancyhdr}   % For custom headers/footers
\usepackage{tabularx}
\usepackage{float}  
% -------------------------------
% Header & Footer setup
% -------------------------------
\pagestyle{fancy}
\fancyhf{} % Clear all header and footer fields

% ULAB logo on the right side of header
\fancyhead[R]{\includegraphics[height=1.2cm]{CommonAssets/ULAB Logo.PNG}}  

% Adjust header spacing
\setlength{\headsep}{1.5cm}

% Page number on the right footer
\fancyfoot[R]{Page \thepage}

% -------------------------------






\title{\textbf{Project Report : Agriculture Crop Yield}\\
STA 2101: Statistics \& Probability}
\author{Student Name : Mahinur Rahman Mahin \\
Student ID : 242014165 \\
University of Liberal Arts Bangladesh (ULAB)}
\date{\today}

\begin{document}

\maketitle
\onehalfspacing

\begin{abstract}
This document is the course project report for STA 2101.This project analyzes by the link of "Agriculture crope Yield".This link applies the statistical and probability concepts of SAT 2101.Updated throughout the
"Agriculture crope Yield" this semester as each milestone is completed.
\end{abstract}

\tableofcontents
\newpage










% -------------------------------


\section{Milestone 1: Dataset Selection}
\begin{itemize}
    \item \textbf{Dataset Name: Agriculture Crop Yield} %[Insert dataset name here]
    \item \textbf{Dataset URL:} \\                \url{https://www.kaggle.com/datasets/samuelotiattakorah/agriculture-crop-yield}  
    \item \textbf{Description:} Rice is the primary food for half of the people in the world. It is also known as staple food in Bangladesh. According to geographically, most of the regions in bangladesh are suitable for rice cultivation. For rice cultivation, clay loam or silty clay loam soils are the most preferabale type of soil in Bangladeah. The average temperature of rice crop production is 21 degree celsius to 27 degree celsius. Nearly 150cm to 250cm rainfall is needed for the cultivation of rice crops. \\\
    Fertilizer and irrigation are used in rice production. \\\
    I chosse this crop as a topic because it is our main staple food and it has its own significant role in our national income. 
\end{itemize}

% -------------------------------












\\\newline
\\\newline
\section{Milestone 2: Descriptive Statistics}
Describe the summary statistics of my dataset. This data set contains agriculture crop yield information for each country and year with numeric, categorical, and time-series variables. The agriculture crop yield averages around 3.6 tons/ha, rainfall 820 mm, and temperature 25–26 degree celsius. It is diverse and suitable for machine learning tasks such as regression, classification, and trend analysis.

Example of a table:

% \begin{table}[h!]
% \centering
% \begin{tabular}{lrr}
% \toprule
% Variable & Mean & Standard Deviation \\
% \midrule
% Column A & 12.3 & 2.1 \\
% Column B & 45.7 & 5.6 \\
% \bottomrule
% \end{tabular}
% \caption{Sample descriptive statistics}
% \end{table}
\begin{table}[h!]
\centering
\scriptsize
\setlength{\tabcolsep}{4pt}
\renewcommand{\arraystretch}{1.1}
\caption{Sample Crop Dataset}
\begin{tabularx}{\textwidth}{|X|X|X|X|X|X|X|X|X|X|}
\hline
\textbf{Region} & \textbf{Soil Type} & \textbf{Crop} & \textbf{Rainfall (mm)} & \textbf{Temp (°C)} & \textbf{Fertilizer Used} & \textbf{Irrigation Used} & \textbf{Weather} & \textbf{Days to Harvest} & \textbf{Yield (t/ha)} \\ 
\hline
North & Sandy & Cotton & 897.07 & 27.67 & False & True & Cloudy & 122 & 6.55 \\ 
\hline
South & Clay & Rice & 992.67 & 18.02 & True & True & Rainy & 140 & 8.52 \\ 
\hline
North & Loam & Barley & 147.99 & 29.79 & False & False & Sunny & 106 & 1.12 \\ 
\hline
North & Sandy & Soybean & 986.86 & 16.64 & False & True & Rainy & 146 & 6.51 \\ 
\hline
South & Silt & Wheat & 730.38 & 31.62 & True & True & Cloudy & 110 & 7.25 \\ 
\hline
South & Silt & Soybean & 797.47 & 37.70 & False & True & Rainy & 74 & 5.89 \\ 
\hline
West & Clay & Wheat & 357.90 & 31.59 & False & False & Rainy & 90 & 2.65 \\ 
\hline
South & Sandy & Rice & 441.13 & 30.89 & True & True & Sunny & 61 & 5.83 \\ 
\hline
\end{tabularx}
\end{table}

\newpage
\section{Part 0 : Probability Sampling Methods}

\begin{figure}[H]
  \centering
  \includegraphics[width=1.0\textwidth]{Screenshot 2025-10-15 001319.png}
  \caption{Overview of Probability Sampling Methods}
  \label{fig:prob_sampling}
  \end{figure}


\subsection*{Part A --- Setup}
\begin{figure}[H]
  \centering
  \includegraphics[width=1.0\textwidth]{Screenshot 2025-10-15 010845.PNG}
  \caption{Setup}
\end{figure}


\\\\


\subsection*{Part B --- Simple Random Sampling}
\begin{figure}[H]
  \centering
  \includegraphics[width=1.0\textwidth]{Screenshot 2025-10-15 011203.png}
  \caption{Simple Random Sampling}
\end{figure}

\subsection*{Part C --- Systematic Sampling}
\begin{figure}[H]
  \centering
  \includegraphics[width=1.0\textwidth]{Screenshot 2025-10-15 011504.png}
  \caption{Systematic Sampling}
\end{figure}

\subsection*{Part D --- Stratified Sampling}
\begin{figure}[H]
  \centering
  \includegraphics[width=1.0\textwidth]{Screenshot 2025-10-15 011751.png}
  \caption{Stratified Sampling}
\end{figure}

\subsection*{Part E --- Cluster Sampling}
\begin{figure}[H]
  \centering
  \includegraphics[width=1.0\textwidth]{Screenshot 2025-10-15 011909.png}
  \caption{Cluster Sampling}
\end{figure}

\subsection*{Part F --- Comparison \& Reflection}
\begin{figure}[H]
  \centering
  \includegraphics[width=0.9\textwidth] {Screenshot 2025-10-15 012244.png} 
  \\
  \includegraphics[width=0.9\textwidth]{Screenshot 2025-10-15 012316.png}
  \caption{Comparison and Reflection}


  
\end{figure}

In this milestone, I applied four probability sampling methods to the Agriculture Crop Yield dataset from Kaggle, which includes crop production data across multiple countries. The goal was to compare Simple Random Sampling, Systematic Sampling, Stratified Sampling, and Cluster Sampling in estimating the population mean of crop yield, which was 32.337344 t/ha.

Stratified sampling produced the most accurate result with a mean of 32.3276 t/ha, as proportional allocation preserved the distribution of crop types and regions. Simple Random Sampling yielded 32.25 t/ha, slightly lower, while systematic sampling gave 32.3872 t/ha, slightly higher. Cluster sampling showed the largest deviation at 32.5075 t/ha due to potential homogeneity within clusters.

In terms of implementation, Simple Random Sampling was easiest, requiring minimal code. Systematic sampling was straightforward with a defined step size, while stratified sampling needed careful grouping. Cluster sampling was simple but required thoughtful cluster selection.

Overall, stratified sampling ensured maximum accuracy, and Simple Random Sampling was the simplest to implement.


% ---------- End Milestone 2 block ----------
% -------------------------------














\newpage
\section{Milestone 3: Data Visualization}
Add graphs and figures using LaTeX. Example:

\begin{figure}[h!]
\centering
\includegraphics[width=0.7\textwidth]{example-figure.png}
\caption{Sample dataset visualization (replace with your figure)}
\end{figure}







% -------------------------------














\newpage
\section{Milestone 4: Probability Distributions}
Identify probability distributions in your dataset. Perform fitting, plots, and discuss results.

% -------------------------------
\section{Milestone 5: Hypothesis Testing}
State hypotheses, perform tests, and report conclusions.

% -------------------------------
\section{Milestone 6: Regression Analysis}
Fit regression models, explain coefficients, and evaluate model fit.

% -------------------------------
\section{Milestone 7--12: Further Analysis}
Continue documenting each milestone here as instructed in class.

% -------------------------------
\section{Final Conclusion}
Summarize the overall findings of your project. Mention challenges, learning outcomes, and possible future work.

\newpage
\section*{References}
List your references here in proper citation format. If you prefer, you may use BibTeX.

\end{document}
