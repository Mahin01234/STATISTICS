\documentclass[12pt,a4paper]{article}

\usepackage{geometry}
\geometry{margin=1in}
\usepackage{graphicx}   % For figures and logo
\usepackage{hyperref}   % For clickable links
\usepackage{booktabs}   % For better tables
\usepackage{amsmath}    % For equations
\usepackage{setspace}   % Line spacing
\usepackage{enumitem}   % Custom lists
\usepackage{fancyhdr}   % For custom headers/footers
\usepackage{tabularx}
\usepackage{float}  
\usepackage{caption}

% -------------------------------
% Header & Footer setup
% -------------------------------
\pagestyle{fancy}
\fancyhf{} % Clear all header and footer fields

% ULAB logo on the right side of header
\fancyhead[R]{\includegraphics[height=1.2cm]{CommonAssets/ULAB Logo.PNG}}  

% Adjust header spacing
\setlength{\headsep}{1.5cm}

% Page number on the right footer
\fancyfoot[R]{Page \thepage}

% -------------------------------






\title{\textbf{Project Report : Agriculture Crop Yield}\\
STA 2101: Statistics \& Probability}
\author{Student Name : Mahinur Rahman Mahin \\
Student ID : 242014165 \\
University of Liberal Arts Bangladesh (ULAB)}
\date{\today}

\begin{document}

\maketitle
\onehalfspacing

\begin{abstract}
This document is the course project report for STA 2101.This project analyzes by the link of "Agriculture crope Yield".This link applies the statistical and probability concepts of SAT 2101.Updated throughout the
"Agriculture crope Yield" this semester as each milestone is completed.
\end{abstract}

\tableofcontents
\newpage










% -------------------------------


\section{Milestone 1: Dataset Selection}
\begin{itemize}
    \item \textbf{Dataset Name: Agriculture Crop Yield} %[Insert dataset name here]
    \item \textbf{Dataset URL:} \\                \url{https://www.kaggle.com/datasets/samuelotiattakorah/agriculture-crop-yield}  
    \item \textbf{Description:} Rice is the primary food for half of the people in the world. It is also known as staple food in Bangladesh. According to geographically, most of the regions in bangladesh are suitable for rice cultivation. For rice cultivation, clay loam or silty clay loam soils are the most preferabale type of soil in Bangladeah. The average temperature of rice crop production is 21 degree celsius to 27 degree celsius. Nearly 150cm to 250cm rainfall is needed for the cultivation of rice crops. \\\
    Fertilizer and irrigation are used in rice production. \\\
    I chosse this crop as a topic because it is our main staple food and it has its own significant role in our national income. 
\end{itemize}

% -------------------------------


\section{Milestone 2: Descriptive Statistics}
Describe the summary statistics of my dataset. This data set contains agriculture crop yield information for each country and year with numeric, categorical, and time-series variables. The agriculture crop yield averages around 3.6 tons/ha, rainfall 820 mm, and temperature 25–26 degree celsius. It is diverse and suitable for machine learning tasks such as regression, classification, and trend analysis.

Example of a table:

% \begin{table}[h!]
% \centering
% \begin{tabular}{lrr}
% \toprule
% Variable & Mean & Standard Deviation \\
% \midrule
% Column A & 12.3 & 2.1 \\
% Column B & 45.7 & 5.6 \\
% \bottomrule
% \end{tabular}
% \caption{Sample descriptive statistics}
% \end{table}
\begin{table}[h!]
\centering
\scriptsize
\setlength{\tabcolsep}{4pt}
\renewcommand{\arraystretch}{1.1}
\caption{Sample Crop Dataset}
\begin{tabularx}{\textwidth}{|X|X|X|X|X|X|X|X|X|X|}
\hline
\textbf{Region} & \textbf{Soil Type} & \textbf{Crop} & \textbf{Rainfall (mm)} & \textbf{Temp (°C)} & \textbf{Fertilizer Used} & \textbf{Irrigation Used} & \textbf{Weather} & \textbf{Days to Harvest} & \textbf{Yield (t/ha)} \\ 
\hline
North & Sandy & Cotton & 897.07 & 27.67 & False & True & Cloudy & 122 & 6.55 \\ 
\hline
South & Clay & Rice & 992.67 & 18.02 & True & True & Rainy & 140 & 8.52 \\ 
\hline
North & Loam & Barley & 147.99 & 29.79 & False & False & Sunny & 106 & 1.12 \\ 
\hline
North & Sandy & Soybean & 986.86 & 16.64 & False & True & Rainy & 146 & 6.51 \\ 
\hline
South & Silt & Wheat & 730.38 & 31.62 & True & True & Cloudy & 110 & 7.25 \\ 
\hline
South & Silt & Soybean & 797.47 & 37.70 & False & True & Rainy & 74 & 5.89 \\ 
\hline
West & Clay & Wheat & 357.90 & 31.59 & False & False & Rainy & 90 & 2.65 \\ 
\hline
South & Sandy & Rice & 441.13 & 30.89 & True & True & Sunny & 61 & 5.83 \\ 
\hline
\end{tabularx}
\end{table}

\newpage
\section{Part 0 : Probability Sampling Methods}

\begin{figure}[H]
  \centering
  \includegraphics[width=1.0\textwidth]{Screenshot 2025-10-15 001319.png}
  \caption{Overview of Probability Sampling Methods}
  \label{fig:prob_sampling}
  \end{figure}


\subsection*{Part A --- Setup}
\begin{figure}[H]
  \centering
  \includegraphics[width=1.0\textwidth]{Screenshot 2025-10-15 010845.PNG}
  \caption{Setup}
\end{figure}


\\\\


\subsection*{Part B --- Simple Random Sampling}
\begin{figure}[H]
  \centering
  \includegraphics[width=1.0\textwidth]{Screenshot 2025-10-15 011203.png}
  \caption{Simple Random Sampling}
\end{figure}

\subsection*{Part C --- Systematic Sampling}
\begin{figure}[H]
  \centering
  \includegraphics[width=1.0\textwidth]{Screenshot 2025-10-15 011504.png}
  \caption{Systematic Sampling}
\end{figure}

\subsection*{Part D --- Stratified Sampling}
\begin{figure}[H]
  \centering
  \includegraphics[width=1.0\textwidth]{Screenshot 2025-10-15 011751.png}
  \caption{Stratified Sampling}
\end{figure}

\subsection*{Part E --- Cluster Sampling}
\begin{figure}[H]
  \centering
  \includegraphics[width=1.0\textwidth]{Screenshot 2025-10-15 011909.png}
  \caption{Cluster Sampling}
\end{figure}

\subsection*{Part F --- Comparison \& Reflection}
\begin{figure}[H]
  \centering
  \includegraphics[width=0.9\textwidth] {Screenshot 2025-10-15 012244.png} 
  \\
  \includegraphics[width=0.9\textwidth]{Screenshot 2025-10-15 012316.png}
  \caption{Comparison and Reflection}


  
\end{figure}

In this milestone, I applied four probability sampling methods to the Agriculture Crop Yield dataset from Kaggle, which includes crop production data across multiple countries. The goal was to compare Simple Random Sampling, Systematic Sampling, Stratified Sampling, and Cluster Sampling in estimating the population mean of crop yield, which was 32.337344 t/ha.

Stratified sampling produced the most accurate result with a mean of 32.3276 t/ha, as proportional allocation preserved the distribution of crop types and regions. Simple Random Sampling yielded 32.25 t/ha, slightly lower, while systematic sampling gave 32.3872 t/ha, slightly higher. Cluster sampling showed the largest deviation at 32.5075 t/ha due to potential homogeneity within clusters.

In terms of implementation, Simple Random Sampling was easiest, requiring minimal code. Systematic sampling was straightforward with a defined step size, while stratified sampling needed careful grouping. Cluster sampling was simple but required thoughtful cluster selection.

Overall, stratified sampling ensured maximum accuracy, and Simple Random Sampling was the simplest to implement.


% ---------- End Milestone 2 block ----------
% -------------------------------














\newpage
\section{Milestone 3: Data Visualization}
Add graphs and figures using LaTeX.


\setstretch{1.25}

\pagestyle{fancy}
\fancyhf{} 

\textbf{Implementing Probability Sampling Methods in Python}

\noindent\rule{\textwidth}{0.4pt}


\section*{Part A — Instructions}

\noindent
In this part, the goal is to set up the environment and load the dataset correctly 
before applying different probability sampling techniques. 
The following steps were followed:

\begin{enumerate}
    \item Import necessary Python libraries such as \texttt{pandas}, \texttt{numpy}, and \texttt{IPython.display}.
    \item Load the crop yield dataset using the \texttt{read\_csv()} function.
    \item Display the first few rows of the dataset to verify successful loading.
    \item Calculate the population mean of the \texttt{Yield} column, 
    which serves as the baseline for comparing sampling results.
\end{enumerate}

\noindent
The dataset was successfully loaded, and preliminary statistics were verified before performing sampling.


\end{itemize}

\noindent\rule{\textwidth}{0.4pt}

\section*{Part B - Data Set}
\rhead{Sampling Assignment}
\lhead{Probability Sampling Methods}
\cfoot{\thepage}

\begin{document}

% -------------------------------
\title{\textbf{Sampling Assignment}\\[0.5em]
\large Implementing Probability Sampling Methods in LaTeX}
\author{}
\date{}
\maketitle
% -------------------------------




\begin{verbatim}

Column Name           Description
-------------------------------------------------------------
Region                Geographical region where the crop is grown (North,East,South)
Soil_Type             Type of soil (Clay, Sandy, Loam, Silt, Peaty, Chalky)
Crop                  Type of crop grown (Wheat, Rice, Maize, Barley,Soybean,Cotton)
Rainfall_mm           Amount of rainfall (in millimeters) during crop growth
Temperature_Celsius   Average temperature during crop growth (°C)
Fertilizer_Used       Indicates fertilizer use (True = Yes, False = No)
Irrigation_Used       Indicates irrigation use (True = Yes, False = No)
Weather_Condition     Predominant weather condition (Sunny, Rainy, Cloudy)
Days_to_Harvest       Number of days required for the crop to be harvested
Yield_tons_per_hectare Total yield (in tons per hectare)
\end{verbatim}

\subsection*{�� Summary Statistics}
\begin{verbatim}
Total records: 1,000,000

Regions:
North - 25%
West - 25%
Other - 50%

Soil Types:
Sandy - 17%
Loam - 17%
Other - 66%

Crops:
Maize - 17%
Rice - 17%
Other - 66%
Fertilizer Used: 50% True, 50% False
Irrigation Used: 50% True, 50% False
Weather Condition: 33% Sunny, 33% Rainy, 33% Cloudy
\end{verbatim}


\subsection*{�� Data Records}

\begin{center}
\scriptsize
\begin{adjustbox}{width=\textwidth}
\begin{tabular}{|c|c|c|c|c|c|c|c|c|c|}
\hline
\textbf{Region} & \textbf{Soil Type} & \textbf{Crop} & \textbf{Rainfall (mm)} & \textbf{Temp (°C)} & \textbf{Fert.} & \textbf{Irrig.} & \textbf{Weather} & \textbf{Days} & \textbf{Yield (t/ha)} \\
\hline
West & Sandy & Cotton & 897.08 & 27.68 & False & True & Cloudy & 122 & 6.56 \\
South & Clay & Rice & 992.67 & 18.03 & True & True & Rainy & 140 & 8.53 \\
North & Loam & Barley & 148.00 & 29.79 & False & False & Sunny & 106 & 1.13 \\
North & Sandy & Soybean & 986.87 & 16.64 & False & True & Rainy & 146 & 6.52 \\
South & Silt & Wheat & 730.38 & 31.62 & True & True & Cloudy & 110 & 7.25 \\
\hline
\end{tabular}
\end{adjustbox}
\end{center}

% -------------------------------



\section*{C. Task 1: Frequency Distribution Table}

\noindent
In this task, a frequency distribution table was created to summarize the crop yield dataset. 
The table shows how data values are distributed across different classes or intervals, 
helping to visualize the overall pattern of the dataset.

\begin{center}
\small
\begin{tabular}{|c|c|c|}
\hline
\textbf{Class Interval (Yield)} & \textbf{Frequency (f)} & \textbf{Relative Frequency (\%)} \\
\hline
1.0 -- 2.9 & 3 & 6.0 \\
3.0 -- 4.9 & 7 & 14.0 \\
5.0 -- 6.9 & 20 & 40.0 \\
7.0 -- 8.9 & 15 & 30.0 \\
9.0 -- 10.9 & 5 & 10.0 \\
\hline
\textbf{Total} & \textbf{50} & \textbf{100\%} \\
\hline
\end{tabular}
\end{center}

\noindent
The above table provides an overview of how crop yields are distributed across the given ranges. 
Most yields fall within the 5.0–6.9 and 7.0–8.9 ranges, indicating a concentration of moderate to high productivity.


   
\documentclass[a4paper,12pt]{article}
\usepackage{graphicx}   
\usepackage{float}      
\usepackage{caption}   

\begin{document}
\newpage


\subsection*{Part D. Task 3: Graphical Representation} 





\section*{Data Sample}

\subsection*{First 5 Rows of the 20-Row Sample (Compact Format)}

\begin{table}[htbp]
\centering
\caption{First 5 rows of the 20-row sample from the agricultural dataset}
\label{tab:first_5_rows_compact}
\begin{tabular}{@{}llllccccc@{}}
\toprule
\textbf{Region} & \textbf{Soil} & \textbf{Crop} & \makecell{\textbf{Rain}\\\textbf{(mm)}} & \makecell{\textbf{Temp}\\\textbf{(°C)}} & \makecell{\textbf{Fert.}\\\textbf{Used}} & \makecell{\textbf{Irr.}\\\textbf{Used}} & \textbf{Weather} & \makecell{\textbf{Days to}\\\textbf{Harvest}} & \makecell{\textbf{Yield}\\\textbf{(t/ha)}} \\
\midrule
667236 & Silt & Maize & 347.73 & 39.27 & No & No & Rainy & 138 & 2.69 \\
5647 & Chalky & Wheat & 191.66 & 30.67 & No & No & Cloudy & 113 & 1.06 \\
128429 & Peaty & Rice & 985.49 & 23.66 & No & Yes & Rainy & 95 & 6.98 \\
572477 & Peaty & Rice & 230.49 & 26.07 & No & Yes & Sunny & 96 & 2.89 \\
181467 & Peaty & Barley & 944.24 & 20.10 & Yes & No & Rainy & 147 & 6.19 \\
\bottomrule
\end{tabular}
\end{table}

\subsection*{Alternative: Even More Compact Format}

\begin{table}[htbp]
\centering
\caption{First 5 rows with minimal abbreviations}
\label{tab:first_5_rows_minimal}
\begin{tabular}{@{}llllcccc@{}}
\toprule
\textbf{ID} & \textbf{Region} & \textbf{Soil} & \textbf{Crop} & \textbf{Rain} & \textbf{Temp} & \textbf{F/I} & \textbf{Yield} \\
& & & & \textbf{(mm)} & \textbf{(°C)} & & \textbf{(t/ha)} \\
\midrule
667236 & South & Silt & Maize & 347.73 & 39.27 & N/N & 2.69 \\
5647 & North & Chalky & Wheat & 191.66 & 30.67 & N/N & 1.06 \\
128429 & East & Peaty & Rice & 985.49 & 23.66 & N/Y & 6.98 \\
572477 & West & Peaty & Rice & 230.49 & 26.07 & N/Y & 2.89 \\
181467 & North & Peaty & Barley & 944.24 & 20.10 & Y/N & 6.19 \\
\bottomrule
\end{tabular}
\vspace{0.2cm}
\footnotesize
\textbf{Note:} F/I = Fertilizer Used/Irrigation Used (Y=Yes, N=No)
\end{table}

\subsection*{Vertical Format (Best for Many Columns)}

\begin{table}[htbp]
\centering
\caption{First 5 rows in vertical format}
\label{tab:first_5_rows_vertical}
\begin{minipage}{0.48\textwidth}
\centering
\begin{tabular}{|l|l|}
\hline
\multicolumn{2}{|c|}{\textbf{Row 1}} \\
\hline
Region & 667236 \\
Soil Type & Silt \\
Crop & Maize \\
Rainfall (mm) & 347.73 \\
Temp (°C) & 39.27 \\
Fertilizer Used & No \\
Irrigation Used & No \\
Weather & Rainy \\
Days to Harvest & 138 \\
Yield (t/ha) & 2.69 \\
\hline
\end{tabular}
\end{minipage}
\hfill
\begin{minipage}{0.48\textwidth}
\centering
\begin{tabular}{|l|l|}
\hline
\multicolumn{2}{|c|}{\textbf{Row 2}} \\
\hline
Region & 5647 \\
Soil Type & Chalky \\
Crop & Wheat \\
Rainfall (mm) & 191.66 \\
Temp (°C) & 30.67 \\
Fertilizer Used & No \\
Irrigation Used & No \\
Weather & Cloudy \\
Days to Harvest & 113 \\
Yield (t/ha) & 1.06 \\
\hline
\end{tabular}
\end{minipage}

\vspace{0.5cm}

\begin{minipage}{0.48\textwidth}
\centering
\begin{tabular}{|l|l|}
\hline
\multicolumn{2}{|c|}{\textbf{Row 3}} \\
\hline
Region & 128429 \\
Soil Type & Peaty \\
Crop & Rice \\
Rainfall (mm) & 985.49 \\
Temp (°C) & 23.66 \\
Fertilizer Used & No \\
Irrigation Used & Yes \\
Weather & Rainy \\
Days to Harvest & 95 \\
Yield (t/ha) & 6.98 \\
\hline
\end{tabular}
\end{minipage}
\hfill
\begin{minipage}{0.48\textwidth}
\centering
\begin{tabular}{|l|l|}
\hline
\multicolumn{2}{|c|}{\textbf{Row 4}} \\
\hline
Region & 572477 \\
Soil Type & Peaty \\
Crop & Rice \\
Rainfall (mm) & 230.49 \\
Temp (°C) & 26.07 \\
Fertilizer Used & No \\
Irrigation Used & Yes \\
Weather & Sunny \\
Days to Harvest & 96 \\
Yield (t/ha) & 2.89 \\
\hline
\end{tabular}
\end{minipage}

\vspace{0.5cm}

\centering
\begin{minipage}{0.48\textwidth}
\centering
\begin{tabular}{|l|l|}
\hline
\multicolumn{2}{|c|}{\textbf{Row 5}} \\
\hline
Region & 181467 \\
Soil Type & Peaty \\
Crop & Barley \\
Rainfall (mm) & 944.24 \\
Temp (°C) & 20.10 \\
Fertilizer Used & Yes \\
Irrigation Used & No \\
Weather & Rainy \\
Days to Harvest & 147 \\
Yield (t/ha) & 6.19 \\
\hline
\end{tabular}
\end{minipage}
\end{table}

\subsection*{Data Summary}

\textbf{Key Observations:}
\begin{itemize}
    \item Highest yield: 6.98 t/ha (Rice in East region)
    \item Lowest yield: 1.06 t/ha (Wheat in North region)
    \item Most common soil: Peaty (3 out of 5 rows)
    \item Most common weather: Rainy (3 out of 5 rows)
    \item Fertilizer used in only 1 case
\end{itemize}








\subsection*{Bar Chart of Crop Type Frequency Distribution}
\subsection*{Task 1:Bar Chart} 
\noindent The following Python code was used to generate a bar chart showing the frequency distribution of different crop types in a 20-row sample from the dataset.

\begin{figure}[H]
    \centering
    \includegraphics[width=0.95\textwidth]{Screenshot 2025-12-01 205747.png}
   
    \label{fig:bar_chart_crop_freq}
\end{figure}

\noindent \textbf{Observations:}
\begin{itemize}
    \item \textbf{Wheat} appears most frequently in the sample (6 occurrences), indicating it might be the most commonly cultivated crop in this subset.
    \item \textbf{Soybean} follows with 4 occurrences, showing moderate prevalence.
    \item \textbf{Cotton} and \textbf{Rice} both appear 3 times each.
    \item \textbf{Barley} and \textbf{Maize} appear least frequently with only 2 occurrences each.
    \item The distribution suggests that cereals (Wheat, Rice, Barley) and legumes/oilseeds (Soybean) dominate the sample, while fiber crops (Cotton) and coarse cereals (Maize) are less represented.
\end{itemize}

\begin{table}[H]
    \centering
    \begin{tabular}{|c|c|}
    \hline
    \textbf{Crop Type} & \textbf{Frequency} \\
    \hline
    Wheat & 6 \\
    Soybean & 4 \\
    Cotton & 3 \\
    Rice & 3 \\
    Barley & 2 \\
    Maize & 2 \\
    \hline
    \end{tabular}
    \caption{Frequency distribution of crop types in the 20-row sample}
    \label{tab:crop_freq_table}
\end{table}


\clearpage
\subsection*{Task 2: Histogram of Yield}

\noindent The following Python code was used to generate the Line Chart showing changes in yield over time or across different regions.

\begin{figure}[H]
    \centering
   
    \includegraphics[width=0.9\textwidth]{Screenshot 2025-12-01 210541.png}
    \caption{Line Chart showing Yield Trends}
    \label{fig:line_chart}
\end{figure}




\clearpage
\subsection*{Task 3: Ogive Chart}


\noindent The following Python code was used to generate the Line Chart showing changes in yield over time or across different regions.

\begin{figure}[H]
    \centering
    
    \includegraphics[width=0.9\textwidth]{Screenshot 2025-12-01 213053.png}
    \caption{Line Chart showing Yield Trends}
    \label{fig:line_chart}
\end{figure}




\clearpage
\subsection*{ask 4: Frequency Polygon of Yield}


\noindent The following Python code was used to generate the Line Chart showing changes in yield over time or across different regions.

\begin{figure}[H]
    \centering
    
    \includegraphics[width=0.9\textwidth]{Screenshot 2025-12-01 215123.png}
    \caption{Line Chart showing Yield Trends}
    \label{fig:line_chart}
\end{figure}



\begin{document}

\section*{E. Task 3 : Analysis and Conclusion}



\subsection*{Frequency Table Insights}
\begin{itemize}
    \item The frequency table shows which yield range or category occurs most frequently.
    \item For the column \texttt{Yield\_tons\_per\_hectare}, the most frequent values are around the mid-range of crop yields.
    \item From the relative frequency and cumulative frequency, it is evident that roughly half of the data falls below the median value.
\end{itemize}

\subsection*{Bar Chart (Regional Analysis)}
\begin{itemize}
    \item The Bar chart highlights significant differences in crop yields across regions.
    \item West and South regions tend to have higher yields.
    \item North region shows comparatively lower productivity.
\end{itemize}

\subsection*{Ogive Charts (Cumulative Frequency Analysis)}
\begin{itemize}
    \item The ``Less than'' Ogive chart is roughly S-shaped, indicating that about half of the data falls below the median.
    \item The ``More than'' Ogive chart shows a slower rise at higher yield values, suggesting that a few farms achieve exceptionally high yields.
    \item Ogive charts help in understanding cumulative distribution and make skewness of the data visible.
\end{itemize}

\subsection*{Distribution Shape \& Variability}
\begin{itemize}
    \item Histogram indicates the distribution is approximately symmetric with a slight right skew.
    \item Some high-yield and low-yield observations may be outliers.
    \item Standard deviation indicates moderate to high variability in the data.
\end{itemize}

\subsection*{Conclusion}
\begin{itemize}
    \item Crop yield data roughly follows a normal distribution, with some right skew and a few outliers.
    \item Regional variations are evident, with certain regions consistently achieving higher yields.
    \item Frequency table, Bar chart, and Ogive analysis together provide a clear understanding of distribution patterns, cumulative trends, and regional disparities.
    \item This analysis is useful for agricultural planning and decision-making for targeted interventions.
\end{itemize}
}




\section*{F. Task 4: Challenges} 


\section*{Challenges Faced}

During this milestone, several challenges were encountered while analyzing the Agriculture Crop Yield dataset:

\begin{enumerate}
    \item \textbf{Selecting the Right Column:} \\
    \textbf{Challenge:} The dataset contains multiple variables, making it difficult to choose which column to analyze. \\
    \textbf{Solution:} \texttt{Yield\_tons\_per\_hectare} was chosen because it directly represents crop productivity and is highly relevant for understanding distribution patterns.
    
    \item \textbf{Deciding on Class Intervals:} \\
    \textbf{Challenge:} Determining appropriate class intervals for frequency distribution was tricky due to the wide range of yield values. \\
    \textbf{Solution:} The Square Root Method was used to determine the number of classes and calculate suitable interval widths based on the data range.
    
    \item \textbf{Generating Visualizations:} \\
    \textbf{Challenge:} Selecting the most effective visualization for the data. \\
    \textbf{Solution:} Multiple visualizations were created:
    \begin{itemize}
        \item Histogram – to see the distribution of yield values.
        \item Bar Chart – to compare average yields across regions.
        \item Frequency Polygon – to show smooth distribution patterns.
        \item Ogive Chart – to analyze cumulative frequency and percentiles.
    \end{itemize}
    
    \item \textbf{Data Cleaning and Processing:} \\
    \textbf{Challenge:} The dataset contained missing values and potential outliers that could affect analysis. \\
    \textbf{Solution:} Missing values were filled or handled, and outliers were identified/removed to ensure accurate results.
\end{enumerate}

\noindent \textbf{Conclusion:} \\
Overcoming these challenges allowed a thorough statistical analysis and creation of clear, informative visualizations. It helped in understanding dataset distribution patterns, regional disparities, and overall crop yield characteristics.






































\newpage

\section{Milestone 4: Probability Distributions}
Identify the probability distributions in your dataset. fitting, plot, and discuss the results.





% -------------------------------




\end{document}


\section{Milestone 5: Hypothesis Testing}
State hypotheses, perform tests, and report conclusions.

% -------------------------------
\section{Milestone 6: Regression Analysis}
Fit regression models, explain coefficients, and evaluate model fit.

% -------------------------------
\section{Milestone 7--12: Further Analysis}
Continue documenting each milestone here as instructed in class.

% -------------------------------
\section{Final Conclusion}
Summarize the overall findings of your project. Mention challenges, learning outcomes, and possible future work.

\newpage
\section*{References}
List your references here in proper citation format. If you prefer, you may use BibTeX.

\end{document}
